\documentclass[12pt,a4paper]{article}
\usepackage{etex,datetime,setspace,latexsym,amssymb,amsmath,amsthm,mathtools}
\usepackage{fancybox,dialogue,float,wrapfig,enumerate,enumitem,microtype}
\usepackage{verbatim,multicol,titlesec,tabularx,mdframed}
\usepackage[dvipsnames]{xcolor}

\theoremstyle{definition}
\newtheorem{definition}{Definition}[section]

\usepackage[utf8]{inputenc}
\usepackage[pdftex]{hyperref}
\usepackage[margin=2cm,bottom=3cm,footskip=15mm]{geometry}
\parindent0cm
\parskip1em

\usepackage{tikz}
\usetikzlibrary{arrows,trees,positioning,shapes,patterns}
\usetikzlibrary{intersections,calc,fpu,decorations.pathreplacing}

\usepackage[T1]{fontenc} % better fonts

% Haskell code listings in our own style
\usepackage{listings,color}
\definecolor{lightgrey}{gray}{0.35}
\definecolor{darkgrey}{gray}{0.20}
\definecolor{lightestyellow}{rgb}{1,1,0.92}
\definecolor{green}{rgb}{0,.2,0}
\definecolor{blue}{rgb}{0,0,.2}
\definecolor{dkyellow}{cmyk}{0,0,.7,.5}
\definecolor{lightgrey}{gray}{0.4}
\definecolor{gray}{gray}{0.50}
\lstset{
  language        = Haskell,
  basicstyle      = \scriptsize\ttfamily,
  keywordstyle    = \color{RoyalBlue},     stringstyle     = \color{purple},
  identifierstyle = \color{black},    commentstyle    = \color{gray},
  showspaces      = false,              showstringspaces= false,
  rulecolor       = \color{gray},       showtabs        = false,
  tabsize         = 8,                  breaklines      = true,
  xleftmargin     = 8pt,                xrightmargin    = 8pt,
  frame           = single,             stepnumber      = 1,
  aboveskip       = 2pt plus 1pt,
  belowskip       = 8pt plus 3pt
}
\lstnewenvironment{code}[0]{}{}
\lstnewenvironment{showCode}[0]{\lstset{numbers=none}}{} % only shown, not compiled

% will the real phi please stand up
\newcommand{\fakephi}{\phi}
\renewcommand{\phi}{\varphi}

% load hyperref as late as possible for compatibility
\usepackage[pdftex]{hyperref}
\hypersetup{pdfborder = {0 0 0}, breaklinks = true}



\title{Heuristic-Driven Theory Projection in Haskell}
\author{M. Petrowitch, X. Ripoll, R. Tosswill}
\date{\today}
\hypersetup{pdfauthor={M. Petrowitch, X. Ripoll, R. Tosswill}, pdftitle={Heuristic-Driven Theory Projection in Haskell}}

\begin{document}

\maketitle

\begin{abstract}
In \cite{Schmidt-2014}, the authors provide an overview of Heuristic-Driven
Theory Projection (HDTP), a logic-based computational model of analogical
reasoning. In this framework, an agent's knowledge of a familiar domain $S$ is
represented as a first-order theory, which can be projected into another, less
familiar domain $T$, by constructing a more general domain $G$, using
anti-unification of formulas \cite{Plotkin70}. 

To our knowledge, there has been so far no implementation of HDTP in a functional programming language. In the following report, we provide a functional algorithm in Haskell for reproducing many of the aspects of HDTP laid out in \cite{Schmidt-2014}.
\end{abstract}

\vfill

\tableofcontents



\clearpage

% We include one file for each section. The ones containing code should
% be called something.lhs and also mentioned in the .cabal file.

\section{Analogical Reasoning and HDTP}
Analogical reasoning is a central aspect of human problem-solving skills
\cite{gentner}, creativity \cite{Besold2015GeneralizeAB}, and concept
formation \cite{hofstadter2013surfaces}, and as such, is a major focus of
research in cognitive science. Traditionally, analogical reasoning is defined to be the process in which an agent will attempt to understand an unfamiliar domain of knowledge $T$ (the \textit{target} domain) by noticing its similarity to a more familiar domain $S$ (the \textit{source} domain). This similarity allows an agent to propose novel hypotheses about the target domain, supposing that s/he has discovered a useful similarity between source and target.

The HDTP framework as laid out in \cite{Schmidt-2014}, approaches analogical reasoning from a syntactic point of view, representing domains of knowledge as first-order theories, and analogies as formal correspondences between formulae of those theories. 

HDTP sees the process of analogical reasoning as consisting of three phases: 
\begin{itemize}
    \item \textit{Retrieval}, in which the source domain is formalized as a first-order theory;
    \item \textit{Mapping}, in which a formal correspondence is established between formulae in the source and target domains, by way of \textit{anti-unification};
    \item \textit{Transfer}, in which the correspondence established in the mapping phase is used to generate new formulae in the target domain, to serve as hypotheses in further reasoning.
\end{itemize} 

These three phases will guide our discussion of the current Haskell implementation over the course of the next three sections: Section 2 will provide details about how we have implemented first order theories, which will also serve to cover the retrieval phase of HDTP. Sections 3 and 4 will focus on the implementation of restricted higher-order anti-unification, which is the formal process at the heart of the mapping phase of HDTP. Section 5 will lay out how we have implemented a procedure for analogical transfer. In Section 6, we will summarize the achievements of the project so far, as well as what remains to be completed in further work.
% 
\section{How to use this?}

To generate the PDF, open \texttt{report.tex} in your favorite \LaTeX editor and compile.
Alternatively, you can manually do
\texttt{pdflatex report; bibtex report; pdflatex report; pdflatex report} in a terminal.

You should have stack installed (see \url{https://haskellstack.org/}) and
open a terminal in the same folder.

\begin{itemize}
  \item To compile everything: \verb|stack build|.
  \item To open ghci and play with your code: \verb|stack ghci|
  \item To run the executable from Section \ref{sec:Main}: \verb|stack build && stack exec myprogram|
  \item To run the tests from Section \ref{sec:simpletests}: \verb|stack clean && stack test --coverage|
\end{itemize}


[bit about the benefits of implementation as a way to improve understanding]

\input{lib/HDTP.lhs}

\input{test/Tests.lhs}

\section{Conclusion and Future Work}
% 
\section{Conclusion}\label{sec:Conclusion}

Finally, we can see that \cite{liuWang2013:agentTypesHLPE} is a nice paper.




\addcontentsline{toc}{section}{Bibliography}
\bibliographystyle{alpha}
\bibliography{references.bib}

\end{document}